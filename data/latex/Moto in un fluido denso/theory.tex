\subsection{Cenni preliminari}
La situazione che prendiamo in considerazione è la seguente: una sfera di cui ci è noto diametro e densità si muove all'interno di un fluido con densità conosciuta. Calcolare la velocità \(v\) a cui si muove la sfera nel generico istante di tempo \(t\). Supponiamo che nell'istante \(t_0 = 0\) la pallina si muova con velocità \(v_0\).
\hfill 

\begin{flushleft}
\textbf{Nomenclatura preliminare}
\end{flushleft}
\begin{center}
\begin{tabular}{ | m{5em} | m{1.4cm}| m{10em} | } 
\hline
variabile &  u.d.m. & descrizione \\ 
\hline
\hline
\(\phi_{sfera}\) & m & diametro sfera \\ 
\hline
\(\rho_{sfera}\) & \(kg/m^3\) & densità sfera \\ 
\hline
\(V\) & \(m^3\) & volume sfera \\ 
\hline
\(m\) & \(kg\) & massa sfera \\ 
\hline
\(\rho_{fluido}\) & \(kg/m^3\) & densità fluido \\ 
\hline
\(F_p\) & \(N\) & forza peso \\ 
\hline
\(F_{pf}\) & \(N\) & forza peso nel fluido \\ 
\hline
\(F_a\) & \(N\) & forza di Archimede \\ 
\hline
\(F_s\) & \(N\) & attrito di Stokes \\ 
\hline
\(g\) & \(m/s^2\) & accelerazione gravità \\ 
\hline
\(b\) & \(kg/s\) & coefficiente di attrito \\ 
\hline
\end{tabular}
\end{center}

\begin{flushleft}
\textbf{Peso di un oggetto in un fluido}\\
Usando Newton e il principio di Archimede posso ricavare il peso di un corpo in un fluido.
\end{flushleft}
\begin{equation}\label{eq4}
    \begin{split}
    F_{pf} &= mg - V\rho_{fluido}\;g\\
        &= V\rho_{corpo}g - V\rho_{fluido}\;g\\
        &=Vg(\rho_{sfera} - \rho_{fluido})
    \end{split}
\end{equation}

\subsection{Analisi}
Partiamo con il trovare la risultante delle forze che agiscono sulla nostra sfera che sono forza peso, spinta di Archimede e attrito esercitato dal fluido.
\begin{equation}\label{eq5}
\vec{R} = \vec{F_p} + \vec{F_a} + \vec{F_s}
\end{equation}
Essendo il moto mono-direzionale assegno alla direzione di \(g\) verso positivo\footnote{La direzione verso il basso per intenderci.}. Sostituisco nell'equazione (\ref{eq4}) la (\ref{eq2}) e la (\ref{eq3}) tenendo conto dei versi dei vettori.
\begin{equation}\label{eq6}
    \begin{split}
    R &= F_p - F_a - F_s\\
            &= mg - V\rho_{fluido}\;g -bv
    \end{split}
\end{equation}
Ricordandoci che \(m=V\rho_{sfera}\) e che, per la II legge di Newton \(R = ma\), risolvo l'equazione per a. Uso anche l'equazione (\ref{eq4}).
\begin{equation}\label{eq7}
    \begin{split}
    ma  &= mg - V\rho_{fluido}\;g -bv\\
        &= V\rho_{sfera}\;g - V\rho_{fluido}\;g -bv\\   
        &= Vg(\rho_{sfera} - \rho_{fluido}) -bv\\
        &= F_{pf} - bv\\
    a   &= \frac{F_{pf} - bv}{m}\\
    a   &=\frac{b}{m}\left(\frac{F_{pf}}{b} - v\right)
    \end{split}
\end{equation}
Possiamo notare ora che \([\frac{b}{m}] = \frac{1}{s}\), è dunque una frequenza che chiamo \(\frac{1}{\mathcal{T}}\), e che \([\frac{F_{pf}}{b}] = \frac{m}{s}\) è quindi una velocità che chiamo \(v_l\). Riscrivendo l'equazione (\ref{eq7}) tenendo conto della nuova nomenclatura ottengo una formula molto carina:
\begin{equation}\label{eq8}
    a = \frac{1}{\mathcal{T}}\left(v_l - v\right)
\end{equation}
Ma \(a\) non è altro che la derivata di \(v\) rispetto al tempo, mi trovo dunque di fronte ad un equazione differenziale che sono in grado di risolvere usando gli integrali.
\begin{equation}\label{eq9}
    \begin{split}
    a = \frac{1}{\mathcal{T}}\left(v_l - v\right)\\
    \frac{a}{v_l - v} = \frac{1}{\mathcal{T}}\\
    \int\frac{a}{v_l - v}\,dt  = \int\frac{1}{\mathcal{T}}\,dt\\
    \int\frac{dv}{v_l - v}\,\frac{dt}{dt}  = \int\frac{1}{\mathcal{T}}\,dt\\
    -\log\lvert v_l - v \rvert = \frac{1}{\mathcal{T}}t + C\\
    \log\lvert v_l - v \rvert = -\frac{1}{\mathcal{T}}t + C
    \end{split}
\end{equation}
Supponendo ora che \(v < v_l\) possiamo togliere il valore assoluto e applicare l'esponenziale.
\begin{equation}\label{eq10}
    \begin{split}
    \log(v_l - v) = -\frac{1}{\mathcal{T}}t\\
    v_l - v = e^{-\frac{1}{\mathcal{T}}t + C}\\
    v = v_l - e^{-\frac{1}{\mathcal{T}}t}e^C\\
    v(t) = v_l - C'e^{-\frac{1}{\mathcal{T}}t}\\
    \end{split}
\end{equation}
Abbiamo finalmente ricavato l'equazione della velocità, per trovare \(C'\) dobbiamo utilizzare le condizioni iniziali che si trovano nei cenni preliminari all'inizio di questo capitolo.
\begin{equation}\label{eq11}
    \begin{cases}
    v_0 = v_l - C'\\
    C' = v_0 - v_l
    \end{cases} \Rightarrow v(t) \; \;= v_l - (v_0 - v_l)e^{-\frac{1}{\mathcal{T}}t}
\end{equation}
Se \(v_0 = 0\) trovo la seguente formula e ho finito:
\begin{equation}\label{eq12}
    \begin{split}
    v(t) &= v_l - v_l e^{-\frac{1}{\mathcal{T}}t}\\
         &= v_l (1 -  e^{-\frac{1}{\mathcal{T}}t})
    \end{split}
\end{equation}
